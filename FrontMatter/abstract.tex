A desirable security property that a system may want to achieve is untraceability: the inability to trace a given action back to the principal who performed it. This concept may appear unfeasible in today's world, in which any digital message can be mapped to the physical device it was originated from.

The purpose of this report is to present the Dining Cryptographer Network (DC-Net) security protocol, first proposed by David Chaum in 1988, as possible solution to such problem. Through the implementation of a simulation of this protocol, the thesis sets out to analyse the real-world potential and limitations of a DC-Net, which, as a theoretical concept, offers unconditional untraceability to a message sender who is part of such network.
