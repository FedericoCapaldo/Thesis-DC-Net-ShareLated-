\chapter{Professional Issues}
This chapter addresses the legal, social and ethical issues that arise from the development of this project.

\section{Ethical Concerns}
A fully functioning implementation of the of Dining Cryptographers protocol \textit{can} have a strong impact both legally and socially, due to implications of the existence of a working protocol with unconditional untraceability. However, the software implemented in this project is a simulation of the protocol devised for learning purposes. Despite the software mirrors the internal components of a DC-Net closely, it does not address a number of fundamental issues such as the trust problem that arises from the presence of a server, or the key exchange problem at a protocol level, here solved with https protocol. Therefore, there are no ethical implication deriving from this project.


\section{British Computing Society Code of Conduct \& Code of Good Practice}
The project, throughout all of its phases has been observant of the British Computer Society's Code of Conduct. It is important to follow standards and correct practices in order to demonstrate professionalism of one's work. \newline


\textbf{Public Interest:} No group has been discriminated during the execution of the project on the grounds of sex, sexual orientation, marital status, nationality, colour, race, ethnic origin, religion, age or disability. 

In addition, due credit has been given to others' source code whenever required. In particular, it is to be noted that the project file structure was generated by a boilerplate as reported in the appendix. \newline 

\textbf{Professional Competence and Integrity:} The project has been a great opportunity to grow my professional knowledge and skills throughout the whole year, both technically and academically. 

In addition, recurrent alternative points of views were sought to receive valuable criticism to improve the quality of the project. \newline


\textbf{Duty to Relevant Authority:} professional judgment was used to carry out my professional responsibilities with due care and diligence in accordance with King's College London's requirements. 

In addition, none of the part of the project was developed to gain personal benefits. \newline


\textbf{Duty to Profession:} a continuous effort was made to respect the Code of Conduct to uphold the British Computing Society's reputation. \newline 