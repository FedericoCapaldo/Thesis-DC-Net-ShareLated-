\chapter{Introduction}


To protect people's privacy and freedom of speech, different protocols have been developed in an attempt to provide untraceable communication. Untraceability can be understood as the inability to link an action to the principal who performed it. Among the protocols that fulfill such security goal, the Dining Cryptographers Network (DC-Net) is the strongest one to provide untraceability of the sender and receiver. 

What makes DC-Net particularly interesting is its property of unconditional security: the protocol is invulnerable to any kind crypto-analytic attack, no matter the amount of computational power of an adversary. This property holds in theory very well. However, the protocol is generally deemed impractical. Thus, there are very few attempts to simulate it.

Current simulations of DC-Net do not follow the distributed architecture of the protocol, and they are not usable in the real world. Some of them are inaccessible to the user due to cumbersome setup and have a poor user interface.

This project attempts to simulate the protocol faithfully to the theory. In particular, it tries to preserve sender untraceability at all times. By doing so, I aim to demonstrate some of the difficulties of implementing a perfectly secure protocol.

\section{Motivation}
The drive to create a simulation for this protocol is three-fold: 

\begin{enumerate}
\item Despite being the most secure protocol in terms of untraceability, DC-Net presents limitations for real-world implementations that can be addressed;

\item There is a scarce number of projects aimed at simulating the protocol for academics interested in the topic;

\item This project allows me to further explore my interest in Information Security, an area of Computer Science that fascinates me due to the socio-political implications around personal privacy. Moreover, the core idea of a DC-Net is a simple the XOR logic operation (explained in the main body). There is a certain beauty that arises from the fact that applying a simple concept gives life to a protocol that guarantees unconditional security.
\end{enumerate}


\section{Project Scope}
The project focuses on a simulation of the Dining Cryptographer protocol at a practical level. I concentrate on the protocol's ability to guarantee sender anonymity. 

In addition to the simulation tool, part of the scope of this project is to also analyse the challenges encountered first-hand during the implementation. 

I do not treat the issue of detecting collusion of participants.  Moreover, although I provide a solution to exchange secret keys securely, I do not address this issue at a protocol level.


\section{Solution Objectives}
The project undertaken has the purpose of building a simulation tool to demonstrate the functioning of the Dining Cryptographer protocol to those who would like to acquire a pragmatic understanding of it.

The solution consists of a web application, designed to ease its access and usage, by removing possible installation steps that a native application may have required. Moreover, it employs a client-server architecture. The server entity facilitates the implementation of a real distributed system, and helps the exchange of messages between participants.

Although this is a real-time application so multiple clients can communicate instantaneously, delays are intentionally introduced in the execution of the protocol to help the users understand how the protocol works step-by-step. 

In line with this educational objective of the simulation tool, I design a GUI so that the simulator is user-friendly. In this way, the user should find the application instructive in the process of deepening their knowledge of a DC-Net.


\section{Target Audience}
The intended audience of this project is academics and Information Security students interested in the workings of the DC-Net protocol. Overall, I assume the target audience to be highly computer literate at university level.


\section{Personal Contributions}
With the delivery of this project, my contributions to the academic discussion on the DC-Net topic are listed below. \newline

Firstly, the simulation tool implemented is unique in its kind as, to my knowledge, there is no other user-friendly interface application that simulates this theoretical protocol. Consequently, I do not use a command-line tool as this does not offer an intuitive UI.  Moreover, what renders it unique is that I actually deploy the tool as a distributed system rather than simulating the network on a single machine. The simulation is available online at \url{https://dc-net.herokuapp.com}. \newline

Secondly, the experience of building this simulator from scratch has been an authentic journey of self-learning. Thanks to this, I have observed first-hand some of the complex challenges of translating a theoretical security protocol into a deployed piece of work.
Put together, my analysis on the work performed represent a complementary contribution to the academic debate on the DC-Net protocol, the latter focusing entirely on theoretical considerations. \newline


Lastly, given the lack of examples on how to implement a DC-Net, I put forward specific solutions that aim to further bridge the gap between theory and real-world usage of the protocol. Specifically, I propose to interpret round messages with Extended ASCII encoding standard, which enables the transmission of human readable messages. I also introduce the concept of length-calculation round as an intermediate step to voting and communication rounds to preserve untraceability when a message is communicated across multiple rounds. Finally, I try to guarantee a degree of fairness in the network usage by preventing the same user from sending back-to-back.


\section{Report Structure}

The report consists of 8 chapters and its content is presented as it follows:

In chapter 1, the current chapter, I outline the motivation, scope and objectives of the simulation tool.

In chapter 2, I review the differences between anonymity and untraceability, which sets the scene for an in-depth presentation of the DC-Net protocol. I, then, explain the limitations and possible extensions of the basic protocol and review the current state of art of the DC-Net implementations.

In chapter 3, I present the basic and advanced requirements of the application. Following, I present how each requirement is fulfilled in the implementation. 

In chapter 4, I motivate the design choices taken and present the design of the implemented simulation by employing UML diagram notations.

In chapter 5, I delves into the actual implementation. I describe the technologies used, and provide code snippets of the main functionalities. I then outline the challenges faced throughout development and conclude with the mapping of the theoretical concepts to the corresponding features implemented.

In chapter 6, I briefly state how the piece of software implemented does not present ethical issues.

In chapter 7, I evaluate the work undertaken meets the project's requirements, I compared the work against available simulation tools and address the known limitations. I then present my analysis of the difficulties of preserving untraceability when trying to implement a Dining Cryptographer Network.

Lastly, in chapter 8, I summarise the overall project and draw the conclusion over the accomplished work and discusses possible future works.