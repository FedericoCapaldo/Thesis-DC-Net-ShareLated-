\chapter{Introduction}


To protect people's privacy and freedom of speech, different protocols have been developed in an attempt to provide untraceable communication. Untraceability can be understood as the inability to link an action to the principal who performed it.


Among the protocols that fulfill such security goal, the Dining Cryptographers Network (DC-Net) is the strongest one to provide untraceability of the sender and receiver. 

What makes DC-Net particularly interesting is its property of unconditional security: the protocol is invulnerable to any kind crypto-analytic attack, no matter the amount of computational power of an adversary. This property holds in theory very well. However, the protocol is generally deemed impractical. Thus, there are very few attempts to simulate it.

Current simulations of DC-Net do not follow the distributed architecture of the protocol, and they are not usable in the real world. Some of them are inaccessible to the user due to cumbersome setup and have a poor user interface.

This project attempts to simulate the protocol faithfully to the theory. In particular, it tries to preserve sender untraceability at all times. By doing so, I aim to demonstrate some of the difficulties of implementing a perfectly secure protocol.

\section{Motivation}
The drive to create a simulation for this protocol is three-fold: 

\begin{enumerate}
\item Despite being the most secure protocol in terms of untraceability, DC-Net presents limitations for real-world implementations that can be addressed;

\item There is a scarce number of projects aimed at simulating the protocol for academics interested in the topic;

\item This project allows me to further explore my interest in Information Security, an area of Computer Science that fascinates me due to the socio-political implications around personal privacy. Moreover, the core idea of a DC-Net is a simple the XOR logic operation (explained in the main body). There is a certain beauty that arises from the fact that applying a simple concept gives life to a protocol that guarantees unconditional security.
\end{enumerate}


\section{Project Scope}
The project focuses on a simulation of the Dining Cryptographer protocol at a practical level. I concentrate on the protocol's ability to guarantee sender anonymity. 

In addition to the simulation tool, part of the scope of this project is to also analyse the challenges encountered first-hand during the implementation. 

I do not treat the issue of detecting collusion of participants.  Moreover, although I provide a solution to exchange secret keys securely, I do not address this issue at a protocol level.


\section{Solution Objectives}
The project undertaken has the purpose of building a simulation tool to demonstrate the functioning of the Dining Cryptographer protocol to those who would like to acquire a pragmatic understanding of it.

The solution consists of a web application, designed to ease its access and usage, by removing possible installation steps that a native application may have required. Moreover, it employs a client-server architecture. The server entity facilitates the implementation of a real distributed system, and helps the exchange of messages between participants.

Although this is a real-time application so multiple clients can communicate instantaneously, delays are intentionally introduced in the execution of the protocol to help the users understand how the protocol works step-by-step. 

In line with this educational objective of the simulation tool, I design a GUI so that the simulator is user-friendly. In this way, the user should find the application instructive in the process of deepening their knowledge of a DC-Net.


\section{Target Audience}
The intended audience of this project is academics and Information Security students interested in the workings of the DC-Net protocol. Overall, I assume the target audience to be highly computer literate at university level.


\section{Contributions of my thesis}

The 


This has been an authentic journey of discovery it's not by reading other theoretical accounts that i came up with things. it's actually the product of doing the code 



- the simulation tool
- the analysis of moving from theory to practice
- the personal proposal for real world protocol usage (Extended ASCII interpretation; division of communication in voting, length-calculation, communication rounds, preventing user to win next round for network usage fairness)



\section{Report Structure}

The report comprises 8 chapters. The content is divided as it follows:

Chapter 1, the current chapter, outlines the motivation, scope and objectives of the simulator tool.

Chapter 2 introduces background and related work in terms of anonymity and untraceability, which serves as a foreword for the actual in-depth presentation of the DC-Net protocol. A complete proof of security of the protocol is provided, and lastly the current state of art of the DC-Net implementations is looked at.

Chapter 3 explores the requirements, originated in line with the ultimate goals of the project of reproducing a working version of the protocol. Following, there is a presentation of how each of these requirements is going to be fulfilled. 

Chapter 4 examines the design of the architecture from different perspectives by employing diagram notations (e.g. class diagram, data flow diagram). The graphical user interface is also presented here.

Chapter 5 delves into the actual implementation achieved, describing in detail the technologies used, the challenges faced through development iterations, the programming design choices employed to guarantee the security and untraceability of the application, and the mapping between the feature implemented and the theoretical concepts of the protocol.

Chapter 6 briefly touches on the ethical issues related to the piece of software implemented, also mentioning how the projects abides by the rule of the British Computing Society.

Chapter 7 evaluates the work undertaken addressing the known limitations and suggestion future improvements that could be implemented, if more time was available.

Lastly, Chapter 8 summarizes the overall project and draw the conclusion over the accomplished work and discusses possible future works.